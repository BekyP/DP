
\newpage
\section{Zhrnutie}

Vypracovaná druhá časť diplomového projektu pokračuje v nasadenom pláne z prvej časti a nadväzuje najmä na výsledky z prvotných experimentov. Keďže sme sa tento semester rozhodli nájsť čo najlepší model pozornosti zdola nahor založený na princípe neurónových sietí, väčšina práce sa týkala najmä experimentovania s rôznymi sieťami a ich parametrami. Netreba však zabúdať na problémy s datasetom, kedy nám trvalo celkom dlhý časť nájsť dostatočne vhodný. Nakoniec sme sa rozhodli pre dataset DUT-OMRON, z ktorého sa nám dokopy podarilo získať viac ako 5000 vzoriek pre modely neurónových sietí. Pri tých sme sa po niekoľkých pokusoch rozhodli zamerať na 3 typy a to klasickú konvolučnú neurónovú sieť, autoenkóder a autoenkóder s predtrénovaným model pre detekciu objektov. Práve s posledným spomínaným sme strávili značné množstvo času iba aby sme zistili, že sa jedná o slepú uličku a žiadne relevantné výstupy nedostaneme. Zostávajúce dva modely sme boli schopný natrénovať k predikcii máp výraznosti a po ich porovnaní nie len metrikami vizuálnej pozornosti nám ako lepší vyšiel autoenkóder, s ktorým sme sa rozhodli pracovať aj ďalej v nasledujúcom semestri.

Ako ďalší smer, ktorým by sa mala práca uberať, vidíme postupné dodávanie ďalších informácií o scéne neurónovej sieti. Pre začiatok by to mohli byť informácie o polohe rôznych objektov, neskôr by ale bolo ideálne prejsť aj na ďalšie faktory pozornosti zhora nadol, ako napríklad emočný kontext a iné. Bude však nutné premyslieť, ako takéto informácie pre sieť reprezentovať. Ďalšou časťou práce by malo byť vykonanie vlastného experimentu pre získanie dát s dostatočným množstvom participantov, aby na nich bolo možné vyhodnotiť našu navrhnutú sieť a zároveň sa aj porovnať s už existujúcimi riešeniami. 

\iffalse
Vypracovaná prvá časť diplomového projektu zachytáva analýzu problémovej oblasti vizuálnej pozornosti, analýzu nosnej časti pripravovaného riešenia (neurónových sietí) a detailný popis fungovania a princípov existujúcich riešení, modelov k predikcii pozornosti oboch typov. V aktuálnom stave práca poskytuje dostatočné množstvo informácií k oboznámeniu sa s problematikou a pochopenie kľúčových existujúcich riešení. 

Súčasťou odovzdávanej časti je taktiež prvotný návrh neurónovej siete spolu s počiatočnými experimentami a výsledkami. Z nich vyplynulo, že aj keď neurónová sieť tohto typu je schopná naučiť sa isté závislosti v častiach obrázkov a niečo aj predikovať, presnosť sa nejaví byť dostatočná vzhľadom na značné obmedzenie informácií o celej scéne. Tieto prvotné experimenty nám taktiež napovedajú, kam by sa práca mohla v ďalších mesiacoch uberať. V prvom rade vzhľadom na problémy s vyhodnotením predikcií v častiach obrázkov metrikami sa javí ako rozumné upustiť od predikovania na podobrázkoch a sústrediť sa radšej na plnohodnotnú predikciu pre celý obrázok. Ako druhú kľúčovú vec je vhodné uviesť, že samotná konvolučná neurónová sieť pravdepodobne nebude schopná odviesť si všetky závislosti vzhľadom na sémantický kontext scény a pozornosť zhora nadol (top-down). Preto bude prospešné použiť ju v kombinácii napríklad s LSTM sieťou (alebo jej variantami), ktorá vzhľadom na svoju schopnosť "pamäte" \ poskytuje oveľa väčšie možnosti obsiahnutia kľúčových informácií o scéne, čo koniec koncov vyplýva aj z analyzovaných riešení.
\fi
\newpage
\null
\thispagestyle{empty}
\newpage