
\newpage
\section{Zhrnutie}

Vypracovaný diplomový projekt k problematike predikcie vizuálnej pozornosti sa venuje najmä jej časti berúcej do úvahy sémantický kontext scény s názvom pozornosť zhora nadol (z angl. top-down). Po preštudovaní uvedenej oblasti, dostupných datasetov  a existujúcich riešení, s dôrazom na najnovšie prístupy využívajúce strojové učenie a neurónové siete, sme navrhli sériu modelov, s ktorými sme neskôr experimentovali.

V prvotných experimentoch sme sa snažili využiť rôzne kombinácie datasetov, nakoniec nám ale ako najlepšie vyšlo použiť iba jeden, z dôvodu prílišnej rôznorodosti dát v nich (rôzne veľkosti obrázkov, zozbierané počas rôznych experimentov, atď.). Naša voľba pôvodne padla na DUT-OMRON, neskôr sme ale skončili pri datasete SALICON - ten je rozsiahlejší a o niečo lepšie spracovaný. Na ňom sme trénovali navrhnuté modely, najlepšie z nich vyšiel koncept kombinácie autoenkóderu s konvolučnými vrstvami VGG16 siete pre detekciu objektov, kedy práve časť VGG16 siete fungovala ako samostaný model, ktorého predikcie sa v rámci dekóderu spájali s tými z enkóderu. Týmto spôsobom sa nám podarilo dostať do siete informáciu o pozícii objektov, čo sa prejavilo aj na zlepšení samotných predikcií máp vizuálnej pozornosti. Tie sme nakoniec porovnávali s predikciami dvoch existujúcich modelov vizuálnej pozornosti.

Porovnávanie prebiehalo s modelom pre redukciu sémantických medzier (popísaný v kapitole \ref{semantic_gap}) s modelom SAM (popísaný v kapitole \ref{sam}). Porovnávané boli hodnoty metrík na troch datasetoch, SALICON, MIT1003 a CAT2000. Ako základný model pre naše riešenie sme zvolili ten natrénovaný práve na datasete SALICON, pri čom sme potom postupne dotrénovali sieť na zvyšné dva. Dosiahnuté hodnoty sa vo väčšine prípadov blížili hodnotám spomínaných riešení, ale len zriedkavo boli vyššie. To dokazuje, že zvolený prístup bol správny a predikcie sú konkurencie schopné, stále nie sú však najlepšie možné a je čo vylepšovať. Určite by sa dalo ďalej experimentovať práve s časťou pre detekciu objektov, v prípade možnosti trénovania na väčších grafických kartách by napríklad bolo možné zapojiť VGG16 časť do dekóderu skôr a tým získať väčšie množstvo trénovateľných parametrov, ktoré by potencionálne mohli zachytiť viac závislostí sémantického kontextu. Rovnako by dosť mohol pomôcť vačší dataset, ideálne čo najrozmanitejší (aj vnútorné aj vonkajšie exteriéry, príroda, mesto, ľudia, atď.) - tu by sa dalo uvažovať o vlastných experimentoch, tie sú ale dosť časovo náročné a vyžadujú značný počet participantov.
