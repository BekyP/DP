\newpage

\section{Zhrnutie}

\iffalse 
Vypracovaná prvá časť bakalárskeho projektu obsahuje analýzu nosnej časti pripravovaného riešenia, neurónových sietí. K ich správnemu popisu, princípom a fungovaniu bolo nutné prečítať a naštudovať značné množstvo hlavne odborných článkov. Spolu s popisom existujúcich riešení v danej oblasti práca v aktuálnom stave poskytuje dostatok informácií k oboznámeniu sa s popisovanou problematikou. 

Súčasťou odovzdávanej časti sú taktiež aj popisy prvotných experimentov spolu s návrhom neurónovej siete a datasetom použitým počas v rámci experimentovania.
\newline
\fi

% TODO zistit ci treba nejaky plan na buduci semester

\iffalse
Plán a rozdelenie zvyšnej práce je zhruba nasledovný:

\textbf{Skúškové obdobie}
\begin{itemize}
	\item refaktorizácia a „upratanie" doterajšej práce (hlavne kódu)
	\item implementácia návrhu
	\item mať funkčný prototyp schopný predikcie
	\item otestovať rôzne možnosti učenia (meniť veľkosti batch-ov, rýchlosť,...)
	\item v januári prekonzulovať dosiahnuté výsledky spolu s ďalším postupom
\end{itemize}

\textbf{Letný semester}
\begin{itemize}
	\item refaktorizácia a optimalizácia riešenia
	\item postupné dopisovanie práce a dokumentácie
\end{itemize}

Vyššie uvedený plán je iba orientačný, určite sa bude meniť, či už v závislosti od vzniknutých problémov alebo množstva času nutného k splneniu jednotlivých bodov plánu a vypracovaniu častí práce.
\fi