\newpage
\section{Úvod}

Vizuálna výraznosť zhora nadol (z angl. top-down) je veľmi dôležitou súčasťou vizuálnej pozornosti, nakoľko sa pri nej berie do úvahy sémantický kontext pozorovanej scény. To je niečo, čo zohráva kľúčovú úlohu pri snahe výskumníkov umožniť počítačom vnímať a porozumieť obrázkom rovnakým spôsobom ako ľudia. V posledných rokoch sa ako prelomovým v tejto doméne javí nevídaný pokrok v oblasti umelej inteligencie, konkrétne neurónových sietí. Ich charakteristickým znakom je využitie veľkého množstva hierarchickým vrstiev pre spracovanie nelineárnych informácií. Vďaka tomu sa objavili nové možnosti detekcie salientných častí skúmanej scény na základe jej sémantického kontextu.








