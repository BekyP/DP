\newpage
\null
\thispagestyle{empty}
\newpage
\section{Úvod}
\setcounter{page}{1}

Vizuálna výraznosť zhora nadol (z angl. top-down) je veľmi dôležitou súčasťou vizuálnej pozornosti (popísaná v kapitole \ref{saliency}), nakoľko sa pri nej berie do úvahy sémantický kontext pozorovanej scény. To je niečo, čo zohráva kľúčovú úlohu pri snahe výskumníkov umožniť počítačom vnímať a porozumieť obrázkom rovnakým spôsobom ako ľudia. V posledných rokoch sa ako prelomovým v tejto doméne javí nevídaný pokrok v oblasti umelej inteligencie, konkrétne neurónových sietí, ktorých popis spolu s niekoľkými typmi je spomenutý v kapitole \ref{nn}. Ich charakteristickým znakom je využitie veľkého množstva hierarchických vrstiev pre spracovanie nelineárnych informácií. Vďaka tomu sa objavili nové možnosti detekcie salientných častí skúmanej scény na základe jej sémantického kontextu.

Spomedzi riešení berúcich do úvahy spomínaný sémantický kontext a teda predikujúcich pozornosť zhora nadol (top-down), si v kapitole \ref{saliency_models} predstavíme niekoľko zaujímavých príkladov spolu so zaužívanejšími modelmi k predikcii pozornosti zdola nahor (z angl. bottom-up). K ich evaluácii sa používa značné množstvo metrík, z ktorých napoužívanejšie sú vysvetlené v kapitole \ref{metric}.

Po rozanalyzovaní problémovej oblasti nasleduje v kapitole \ref{design} prvotný návrh nášho riešenia (založený na predošlých kapitolách) vo forme konvolučnej neurónovej siete spolu  s prvotnými experimentami na nej a výsledkami.

\newpage
\null
\thispagestyle{empty}



