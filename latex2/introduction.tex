\newpage
\section{Úvod}
\setcounter{page}{1}

Vizuálna výraznosť zhora nadol (z angl. top-down) je veľmi dôležitou súčasťou vizuálnej pozornosti (popísaná v kapitole \ref{saliency}), nakoľko sa pri nej berie do úvahy sémantický kontext pozorovanej scény. To je niečo, čo zohráva kľúčovú úlohu pri snahe výskumníkov umožniť počítačom vnímať a porozumieť obrázkom rovnakým spôsobom ako ľudia. V posledných rokoch sa ako prelomovým v tejto doméne javí nevídaný pokrok v oblasti umelej inteligencie, konkrétne neurónových sietí, ktorých popis spolu s niekoľkými typmi je spomenutý v kapitole \ref{nn}. Ich charakteristickým znakom je využitie veľkého množstva hierarchických vrstiev pre spracovanie nelineárnych informácií. Vďaka tomu sa objavili nové možnosti detekcie salientných častí skúmanej scény na základe jej sémantického kontextu.

Spomedzi riešení berúcich do úvahy spomínaný sémantický kontext a teda predikujúcich pozornosť zhora nadol (top-down), si v kapitole \ref{saliency_models} predstavíme niekoľko zaujímavých príkladov spolu so zaužívanejšími modelmi k predikcii pozornosti zdola nahor (z angl. bottom-up). Pre ich natrénovanie je nutné mať čo najlepšie spracovaný dataset, najlepšie voľne dostupné sú popísané v kapitole \ref{datasets}. K ohodnoteniu kvality modelov po natrénovaní sa používa značné množstvo metrík, z ktorých napoužívanejšie sú vysvetlené v kapitole \ref{metric}.

Na základe preštudovanej problémovej oblasti sme postupne navrhli niekoľko architektúr neurónových sietí (kapitola \ref{design}), od jednoduchších konvolučných až po zložitejšie kombinujúce niekoľko modelov (aj natrénovaných) určených pre riešenie rôznych úloh do jednej komplexnejšie siete. Na jednotlivé návrhy sme nadviazali experimentami v kapitole \ref{experiments}, kde sú postupne popísané priebehy a dosiahnuté výsledky počas hľadania najlepšieho modelu. Ten sme potom porovnávali s už existujúcimi riešeniami (kapitola \ref{compare_section}) na viacerých datasetoch (kapitola \ref{compare_section}) - náš model bol postupne dotrénovávaný na jednotlivé vybrané datasety, aby sa výsledky čo najviac zlepšili. V závere práce sme už len zhrnuli dosiahnuté výsledky. 

\newpage
\null
\thispagestyle{empty}



