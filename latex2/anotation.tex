\newpage
\cleardoublepage
\thispagestyle{plain}
\begin{center}
\begin{Large}
\textbf{Anotácia} \\
\end{Large}
\end{center}
Slovenská technická univerzita v Bratislave \\
FAKULTA INFORMATIKY A INFORMAČNÝCH TECHNOLÓGIÍ \\
\noindent
Študijný program: \Program \\
\noindent
Autor: \Author \\
{Diplomová práca: }\Title \\
Vedúci práce: \Supervisor \\
\Month\ \Year \\
\noindent
\\

Vizuálna pozornosť zohráva veľmi dôležitú rolu v našom vnímaní sveta. Jej postupné modelovanie je kľúčovým procesom pri snahe výskumníkov umožniť počítačom vnímať a porozumieť pozorovanej scéne podobným spôsobom ako ľudia. Staršie postupy založené na matematických operáciách, s ktorých pomocou prebiehala extrakcia základných čŕt, postupne nahrádzajú modernejšie metódy strojového učenia. Práve v tejto oblasti sa za posledných niekoľko rokov podarilo dosiahnuť obrovský pokrok, najmä vďaka neurónovým sieťam. Ich charakteristickým znakom je využitie veľkého množstva hierarchickým vrstiev pre spracovanie nelineárnych informácií, čo prinieslo obrovské množstvo nových možností pre zachytenie pozornosti a detekciu salientných oblastí scény, berúc do úvahy aj jej sémantický kontext. Vďaka nielen týmto jedinečným vlastnostiam disponujú schopnosťou odvodiť si závislosti medzi pozorovaniami a objektami v scéne. Pri správnom trénovaní vedia práve tieto naučené závislosti aplikovať bez väčších problémov na nové dáta a tak sa čo najviac priblížiť svojimi predikciami reálnej pozornosti človeka.

\newpage
\cleardoublepage
\thispagestyle{plain}
\begin{center}
\begin{Large}
\textbf{Annotation} \\
\end{Large}
\end{center}
Slovak University of Technology Bratislava \\
FACULTY OF INFORMATICS AND INFORMATION TECHNOLOGIES \\
\noindent
Degree Course: Intelligent Software Systems \\
\noindent
Author: \Author \\
{Master thesis: } The modeling of human visual attention using computer vision and artificial intelligence \\
Supervisor: \Supervisor \\
May 2018 \\
\noindent
\\


Visual attention plays very important role in our perception of world. Gradual modeling of visual attention is a key process in researchers efforts to allow computers to perceive and to understand observed scene in a similar way as people. Older methods based on the mathematical operations, that helped extract basic features, are gradually replaced by more sophisticated methods of machine learning. During last couple of years, a huge progress have been achieved exactly in this area, mainly because of neural networks. Their characteristic feature is exploitation of a huge number of hierarchical layers for processing of nonlinear information. This has brought enormous possibilities for capturing of attention and detection of salient areas in scene, taking in account also its semantic context. Because of not only these unique characteristics, neural networks have the ability to deduce dependencies between observations and objects in scene. Using the right training, they can apply these learned dependencies without any big problems to the new data and because of that, approximate their predictions  to the real human attention as much as possible.
